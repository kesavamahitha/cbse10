\begin{enumerate}
\item From a point $Q$, $13 \text{ cm}$ away from the centre of a circle, the length of tangent $PQ$ to the circle is $12 \text{ cm}$. The radius of circle \brak{\text{in $cm$}}\\
\begin{enumerate}
\item $25$\\
\item $\sqrt{313}$\\
\item $5$\\
\item $1$\\
\end{enumerate}
%circles
\item In Figure $1$, $AP$, $AQ$ and $BC$ are tangents to the circle. If $AB = 5 \text{cm}$, $AC = 6 \text{ cm}$ and $BC = 4 \text{ cm}$, then the length of $AP$ \brak{\text{in $cm$}} is\\
\begin{figure} [h]
\centering
\begin{tikzpicture}[>=Stealth]

% Circle
\draw (0,0) circle (1cm);

% Points
\coordinate (A) at (0,3);
\coordinate (B) at (-0.7, 1);
\coordinate (C) at (0.7, 1);
\coordinate (P) at (-0.95, 0.3);
\coordinate (Q) at (0.95, 0.3);

% Tangents and Lines
\draw (A) -- (B) node[midway, left] {};
\draw (A) -- (C) node[midway, right] {};
\draw (B) -- (C) node[midway, below] {};
\draw (B) -- (P);
\draw (C) -- (Q);

% Arrows
\draw[->] (P) -- ($(B)!2cm!(P)$);
\draw[->] (Q) -- ($(C)!2cm!(Q)$);

% Perpendicular lines
\draw ($(P)!0.1cm!90:(B)$) -- ($(P)!0.1cm!-90:(B)$);
\draw ($(Q)!0.1cm!90:(C)$) -- ($(Q)!0.1cm!-90:(C)$);

% Labeling points
\node at (A) [above] {A};
\node at (B) [left] {B};
\node at (C) [right] {C};
\node at (P) [left] {P};
\node at (Q) [right] {Q};
\end{tikzpicture}
\end{figure}
\begin{center}
$\text{Figure } 1$
\end{center}
\begin{enumerate}
\item $7.5$\\
\item $15$\\
\item $10$\\
\item $9$\\
\end{enumerate}
%circles
\item The circumference of a circle is $22 \text{ cm}$. The area of its quadrant \brak{\text{in $cm^2$}}\\
\begin{enumerate}
\item $\frac{77}{2}$\\
\item $\frac{77}{4}$\\
\item $\frac{77}{8}$\\
\item $\frac{77}{16}$\\
\end{enumerate}
\item In Figure $3$, a right triangle $ABC$, circumscribe a circle of radius $r$. If $AB$ and $BC$ are of lengths $8 \text{ cm}$ and $6 \text{ cm}$ respectively, find the value of $r$.\\
\begin{figure}[ht]
\centering
\begin{tikzpicture}[scale=1]
    % Triangle ABC
    \coordinate[label=above:$A$] (A) at (0,8);
    \coordinate[label=below left:$B$] (B) at (0,0);
    \coordinate[label=below:$C$] (C) at (6,0);
    \draw (A) -- (B) -- (C) -- cycle;
    
    % Point (18/5, 16/5)
    \coordinate[label=above:$ $] (P) at (3.6,3.2);
    \fill (P) circle (0pt);
    
    % Line and length r
    \draw[thick] (2,2) -- (3.6,3.2);
    \node[label=below left:$O$,circle,fill,inner sep=1.5pt] (O) at (2,2) {};
    \node[label=above:$r$] at (2.8,2.6) {};
    \draw (O) circle (2cm);
    % Right angle at B
    \draw pic["$ $", draw=black, angle radius=0.4cm, angle eccentricity=1.5] {right angle = C--B--A};
\end{tikzpicture}
\end{figure}
\begin{center}
$\text{Figure } 3$
\end{center}
%Circles
\item Prove that the tangents drawn at the ends of a diameter of a circle are parallel.\\
%circles
\item In Figure $4$, $ABCD$ is a square of side $4 \text{ cm}$. A quadrant of a circle of radius $1 \text{ cm}$ is drawn at each vertex of the square and a circle of diameter $2 \text{ cm}$ is also drawn. Find the area of the shaded region. $\brak{\text{Use } \pi = 3.14}$\\
\begin{figure}[ht]
\centering
\begin{tikzpicture}
    % Draw the square
    \draw[thick] (0,0) rectangle (4,4);
    
    % Draw the circle
    \draw[thick,fill=white] (2,2) circle [radius=1];
    
    % Draw the quadrants
    \draw[thick,fill=white] (0,0) -- (1,0) arc [start angle=0, end angle=90, radius=1] -- cycle;
    \draw[thick,fill=white] (4,0) -- (3,0) arc [start angle=180, end angle=90, radius=1] -- cycle;
    \draw[thick,fill=white] (4,4) -- (3,4) arc [start angle=180, end angle=270, radius=1] -- cycle;
    \draw[thick,fill=white] (0,4) -- (1,4) arc [start angle=360, end angle=270, radius=1] -- cycle;
    
    % Add the hatching pattern with controlled density
    \path[pattern=north east lines, pattern color=black] (0,0) rectangle (4,4);
    
    % Exclude the circle and quadrants from the hatching by overlaying with white fill
    \draw[fill=white] (2,2) circle [radius=1];
    \draw[fill=white] (0,0) -- (1,0) arc [start angle=0, end angle=90, radius=1] -- cycle;
    \draw[fill=white] (4,0) -- (3,0) arc [start angle=180, end angle=90, radius=1] -- cycle;
    \draw[fill=white] (4,4) -- (3,4) arc [start angle=180, end angle=270, radius=1] -- cycle;
    \draw[fill=white] (0,4) -- (1,4) arc [start angle=360, end angle=270, radius=1] -- cycle;
    \node at (0,0) [below left] {A};
    \node at (4,0) [below right] {B};
    \node at (4,4) [above right] {C};
    \node at (0,4) [above left] {D};
\end{tikzpicture}
\end{figure}
\begin{center}
$\text{Figure } 4$
\end{center}
%Circles
\item From a rectangular sheet of paper $ABCD$ with $AB = 40 \text{ cm}$ and $AD = 28 \text{ cm}$, a semi-circular portion with $BC$ as diameter is cut off. Find the area of the remaining paper. $\brak{\text{Use } \pi = \frac{22}{7}}$\\
%Circles
\item In Figure $5$, a circle is inscribed in a triangle $PQR$ with $PQ = 10 \text{ cm}$, $QR = 8 \text{ cm}$ and $PR = 12 \text{ cm}$. Find the lengths of $QM$, $RN$ and $PL$.\\
\begin{figure}[ht]
\centering
\begin{tikzpicture}

% Define points
\coordinate (P) at (1.25,9.92);
\coordinate (Q) at (0,0);
\coordinate (R) at (8,0);
\coordinate (N) at (5.25,4);
\coordinate (L) at (0.33,2.9);
\coordinate (M) at (3,0);

% Draw triangle and lines
\draw[thick] (P) -- (Q) -- (R) -- cycle;
\draw[thick] (5.25,4) -- ++(0,-0.08) -- ++(0,0.3); % Perpendicular line at N
\draw[thick] (0.33,2.9) -- ++(-0.08,0) -- ++(0.2,0); % Horizontal line at (0.33,2.9)
\draw[thick] (3,0) -- ++(0,-0.08) -- ++(0,0.2); % Vertical line at M

% Draw incircle
\draw[thick] (3,2.64575) circle (2.64575cm);

% Labels
\node[above right] at (P) {P};
\node[below left] at (Q) {Q};
\node[below right] at (R) {R};
\node[below] at (M) {M};
\node[left] at (L) {L};
\node[above right] at (N) {N};

\end{tikzpicture}
\end{figure}
\begin{center}
$\text{Figure } 5$
\end{center}
%Circles
\item In Figure $6$, $O$ is the centre of the circle with $AC = 24 \text{ cm}$, $AB = 7 \text{ cm}$ and $\angle BOD = 90\degree$. Find the area of shaded region.$\brak{\text{Use } \pi = 3.14}$\\
%Circles
\item In Figure $7$, find the area of shaded region, if $ABCD$ is a square of side $14 \text{ cm}$ and $APD$ and $BPC$ are semicircles.\\
%circles
\item Prove that the length of tangents drawn from an external point to a circle are equal.\\
%circles
\item In Fig. $1$, the sides $AB$, $BC$ and $CA$ of a triangle $ABC$, touch a circle at $P$, $Q$ and $R$ respectively. If $PA = 4\text{ cm}$, $BP = 3\text{ cm}$ and $AC = 11\text{ cm}$, then the length of $BC$ \brak{\text{in $cm$}} is\\
\begin{enumerate}
\item $11$\\
\item $10$\\
\item $14$\\
\item $15$\\
\end{enumerate}
%circles
\item In Fig.$2$, a circle touches the side $DF$ of $\triangle EDF$ at $H$ and touches $ED$ and $EF$ produced at $K$ and $M$ respectively. If $EK = 9\text{ cm}$, then the perimeter of $\triangle EDF$ \brak{\text{in $cm$}} is\\
\begin{enumerate}
\item $18$\\
\item $13.5$\\
\item $12$\\
\item $9$\\
\end{enumerate}
%circles
\item If the area of a circle is equal to sum of the areas of two circles of diameters $10 \text{ cm}$ and $24 \text{ cm}$, then the diameter of the larger circle \brak{\text{in $cm$}} is\\
\begin{enumerate}
\item $34$\\
\item $26$\\
\item $17$\\
\item $14$\\
\end{enumerate}
%circles
\item The length of shadow of a tower on the plane ground is $\sqrt 3 \text{ m}$ times the height of the tower. The angle of elevation of sun is :\\
\begin{enumerate}
\item $45\degree$\\
\item $30\degree$\\
\item $60\degree$\\
\item $90\degree$\\
\end{enumerate}
%circles
\item If the coordinates of one end of a diameter of a circle are $\brak{2,3}$ and the coordinates of its centre are $\brak{-2,5}$, then the coordinates of the other end of the diameter are:\\
\begin{enumerate}
\item $\brak{-6,7}$\\
\item $\brak{6,-7}$\\
\item $\brak{6,7}$\\
\item $\brak{-6,-7}$\\
\end{enumerate}
%circles
\item Tangents $PA$ and $PB$ are drawn from an external point $P$ to two concentric circles with centre $O$ and radii $8 \text{ cm}$ and $5 \text{ cm}$ respectively, as shown in Fig.$3$. If $AP = 15\text{ cm}$, then find the length of $BP$.\\
%circles
\item In Fig.$4$, an isosceles triangle $ABC$, with $AB = AC$, circumscribe a circle. Prove that the point of contact $P$ bisects the base $BC$.\\
%circles
\item In Fig.$5$, the chord $AB$ of the larger of the two concentric circles, with centre $O$, touches the smaller circle at $C$. Prove that $AC = CB$.\\
%circles
\item In Fig.$6$, $OABC$ is a square of side $7 \text{ cm}$. If $OAPC$ is a quadrant of a circle with centre $O$, then find the area of the shaded region. $\brak{\text{Use } \pi = 3.14}$\\ 

%Circles
\item In Figure $3$, a right triangle $ABC$, circumscribe a circle of radius $r$. If $AB$ and $BC$ are of lengths $8 \text{ cm}$ and $6 \text{ cm}$ respectively, find the value of $r$.\\
\begin{figure}[ht]
\centering
\begin{tikzpicture}[scale=1]
    % Triangle ABC
    \coordinate[label=above:$A$] (A) at (0,8);
    \coordinate[label=below left:$B$] (B) at (0,0);
    \coordinate[label=below:$C$] (C) at (6,0);
    \draw (A) -- (B) -- (C) -- cycle;
    
    % Point (18/5, 16/5)
    \coordinate[label=above:$ $] (P) at (3.6,3.2);
    \fill (P) circle (0pt);
    
    % Line and length r
    \draw[thick] (2,2) -- (3.6,3.2);
    \node[label=below left:$O$,circle,fill,inner sep=1.5pt] (O) at (2,2) {};
    \node[label=above:$r$] at (2.8,2.6) {};
    \draw (O) circle (2cm);
    % Right angle at B
    \draw pic["$ $", draw=black, angle radius=0.4cm, angle eccentricity=1.5] {right angle = C--B--A};
\end{tikzpicture}
\end{figure}
\begin{center}
$\text{Figure } 3$
\end{center}
%Circles
\item Prove that the tangents drawn at the ends of a diameter of a circle are parallel.\\
%circles
\item In Figure $4$, $ABCD$ is a square of side $4 \text{ cm}$. A quadrant of a circle of radius $1 \text{ cm}$ is drawn at each vertex of the square and a circle of diameter $2 \text{ cm}$ is also drawn. Find the area of the shaded region. $\brak{\text{Use } \pi = 3.14}$\\
\begin{figure}[ht]
\centering
\begin{tikzpicture}
    % Draw the square
    \draw[thick] (0,0) rectangle (4,4);
    
    % Draw the circle
    \draw[thick,fill=white] (2,2) circle [radius=1];
    
    % Draw the quadrants
    \draw[thick,fill=white] (0,0) -- (1,0) arc [start angle=0, end angle=90, radius=1] -- cycle;
    \draw[thick,fill=white] (4,0) -- (3,0) arc [start angle=180, end angle=90, radius=1] -- cycle;
    \draw[thick,fill=white] (4,4) -- (3,4) arc [start angle=180, end angle=270, radius=1] -- cycle;
    \draw[thick,fill=white] (0,4) -- (1,4) arc [start angle=360, end angle=270, radius=1] -- cycle;
    
    % Add the hatching pattern with controlled density
    \path[pattern=north east lines, pattern color=black] (0,0) rectangle (4,4);
    
    % Exclude the circle and quadrants from the hatching by overlaying with white fill
    \draw[fill=white] (2,2) circle [radius=1];
    \draw[fill=white] (0,0) -- (1,0) arc [start angle=0, end angle=90, radius=1] -- cycle;
    \draw[fill=white] (4,0) -- (3,0) arc [start angle=180, end angle=90, radius=1] -- cycle;
    \draw[fill=white] (4,4) -- (3,4) arc [start angle=180, end angle=270, radius=1] -- cycle;
    \draw[fill=white] (0,4) -- (1,4) arc [start angle=360, end angle=270, radius=1] -- cycle;
\end{tikzpicture}
\end{figure}
\begin{center}
$\text{Figure } 4$
\end{center}
%Circles
\item From a rectangular sheet of paper $ABCD$ with $AB = 40 \text{ cm}$ and $AD = 28 \text{ cm}$, a semi-circular portion with $BC$ as diameter is cut off. Find the area of the remaining paper. $\brak{\text{Use } \pi = \frac{22}{7}}$\\
%Circles
\item In Figure $5$, a circle is inscribed in a triangle $PQR$ with $PQ = 10 \text{ cm}$, $QR = 8 \text{ cm}$ and $PR = 12 \text{ cm}$. Find the lengths of $QM$, $RN$ and $PL$.\\
%Circles
\item In Figure $6$, $O$ is the centre of the circle with $AC = 24 \text{ cm}$, $AB = 7 \text{ cm}$ and $\angle BOD = 90\degree$. Find the area of shaded region.$\brak{\text{Use } \pi = 3.14}$\\
%Circles
\item In Figure $7$, find the area of shaded region, if $ABCD$ is a square of side $14 \text{ cm}$ and $APD$ and $BPC$ are semicircles.\\
%circles
\item Prove that the length of tangents drawn from an external point to a circle are equal.\\
%circles
\item In Fig. $1$, the sides $AB$, $BC$ and $CA$ of a triangle $ABC$, touch a circle at $P$, $Q$ and $R$ respectively. If $PA = 4\text{ cm}$, $BP = 3\text{ cm}$ and $AC = 11\text{ cm}$, then the length of $BC$ \brak{\text{in $cm$}} is\\
\begin{enumerate}
\item $11$\\
\item $10$\\
\item $14$\\
\item $15$\\
\end{enumerate}
%circles
\item In Fig.$2$, a circle touches the side $DF$ of $\triangle EDF$ at $H$ and touches $ED$ and $EF$ produced at $K$ and $M$ respectively. If $EK = 9\text{ cm}$, then the perimeter of $\triangle EDF$ \brak{\text{in $cm$}} is\\
\begin{enumerate}
\item $18$\\
\item $13.5$\\
\item $12$\\
\item $9$\\
\end{enumerate}
%circles
\item If the area of a circle is equal to sum of the areas of two circles of diameters $10 \text{ cm}$ and $24 \text{ cm}$, then the diameter of the larger circle \brak{\text{in $cm$}} is\\
\begin{enumerate}
\item $34$\\
\item $26$\\
\item $17$\\
\item $14$\\
\end{enumerate}
%circles
\item The length of shadow of a tower on the plane ground is $\sqrt 3 \text{ m}$ times the height of the tower. The angle of elevation of sun is :\\
\begin{enumerate}
\item $45\degree$\\
\item $30\degree$\\
\item $60\degree$\\
\item $90\degree$\\
\end{enumerate}
%circles
\item If the coordinates of one end of a diameter of a circle are $\brak{2,3}$ and the coordinates of its centre are $\brak{-2,5}$, then the coordinates of the other end of the diameter are:\\
\begin{enumerate}
\item $\brak{-6,7}$\\
\item $\brak{6,-7}$\\
\item $\brak{6,7}$\\
\item $\brak{-6,-7}$\\
\end{enumerate}
%circles
\item Tangents $PA$ and $PB$ are drawn from an external point $P$ to two concentric circles with centre $O$ and radii $8 \text{ cm}$ and $5 \text{ cm}$ respectively, as shown in Fig.$3$. If $AP = 15\text{ cm}$, then find the length of $BP$.\\
%circles
\item In Fig.$4$, an isosceles triangle $ABC$, with $AB = AC$, circumscribe a circle. Prove that the point of contact $P$ bisects the base $BC$.\\
%circles
\item In Fig.$5$, the chord $AB$ of the larger of the two concentric circles, with centre $O$, touches the smaller circle at $C$. Prove that $AC = CB$.\\
%circles
\item In Fig.$6$, $OABC$ is a square of side $7 \text{ cm}$. If $OAPC$ is a quadrant of a circle with centre $O$, then find the area of the shaded region. $\brak{\text{Use } \pi = 3.14}$\\ 
%circles
\item Prove that the parallelogram circumscribing a circle is rhombus.\\
%circles
\item Prove that opposite sides of a quadrilateral circumscribing a circle subtend supplementary angles at the centre of the circle.\\
%circles
\item In Fig.$7$, $PQ$ and $AB$ are respectively the arcs of two concentric circles of radii $7\text{ cm}$ and $3.5\text{ cm}$ and centre $O$. If $\angle POQ = 30\degree$, then find the area of the shaded region. $\brak{\text{Use } \pi = \frac{22}{7}}$\\
\item $AB + CD = AD + BC$\\
Prove that the tangent at any point of a circle is perpendicular to the radius through the point of contact.\\

\item A quadrilateral $ABCD$ is drawn to circumscribe a circle. Prove that\\
$AB + CD = AD + BC$\\
\item In Figure $1$, $PQ$ and $PR$ are tangents to a circle with centre $A$. If $\angle QPA = 27\degree$, then $\angle QAR$ equals\\
\begin{enumerate}
\item $63\degree$\\
\item $153\degree$\\
\item $126\degree$\\
\item $117\degree$\\
\end{enumerate}
\item In Figure $2$, $AB$ and $AC$ are tangents to a circle with centre $O$ and radius $8\text{ cm}$. If $OA = 17\text{ cm}$, then the length of $AC \brak{\text { in cm}}$  is\\
\begin{enumerate}
\item $\sqrt {353}$\\
\item $15$\\
\item $9$\\
\item $25$\\
\end{enumerate}
\item In Figure $3$, three sectors of a circle of radius $7\text{ cm}$, making angles of $60\degree$, $80\degree$, $40\degree$ at the centre are shaded. The area of the shaded region $\brak{in\text{ cm}^2}$ is $\brak{\text{Using } \pi = \frac{22}{7}}$\\
\begin{enumerate}
\item $77$\\
\item $154$\\
\item $44$\\
\item $22$\\
\end{enumerate}
\item The incircle of an isosceles triangle $ABC$, with $AB = AC$, touches the sides $AB$, $BC$ and $CA$ at $D$, $E$ and $F$ respectively. Prove that $E$ bisects $BC$.\\

\item Prove that in two concentric circles. the chord of the larger circle, which touches the smaller circle, is bisected at the point of contact.\\

\item In Figure $4$, the shape of the top of the table is that of a sector of a circle with centre $O$ and $\angle OAB = 90\degree$. If $AO = OB = 42\text{ cm}$, then find the perimeter of the top of the table.\\

\item Find the area of the shaded region in Figure $5$, if $ABCD$ is a square of side $28\text{ cm}$ and $APD$ and $BPC$ are semicircles.\\
\item Two tangents $TP$ and $TQ$ are drawn to a circle with centre $O$ from an external point $T$. Prove that $\angle TPQ = 2\angle OPQ$.\\

\item In Figure $6$, $XY$ and $X'Y'$ are two parallel tangents to a circle with centre $O$ and another tangent $AB$ with point of contact $C$ intersects $XY$ at $A$ and $X'Y'$ at $B$. Prove that $\angle AOB = 90\degree$\\
\item In Figure $7$, $ABCD$ is a square of side $7\text{ cm}$. $DBPA$ and $DQBC$ are quadrants of circles, each of radius $7\text{ cm}$. Find the area of the shaded region.$\brak{\text{Use } \pi = \frac{22}{7}}$\\
\item The length of the minute hand of a clock is $14\text{ cm}$. Find the area swept by the minute hand in $10$ minutes. $\brak{\text{Use } \pi = \frac{22}{7}}$\\
\item Prove that the tangent at any point of a circle is perpendicular to the radius through the point of contact.\\
\end{enumerate}
